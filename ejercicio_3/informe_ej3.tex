\documentclass[a4paper,12pt]{article}
\usepackage[left=2cm, right=2cm, top=2cm]{geometry}
\usepackage{karnaugh-map}
\usepackage[utf8]{inputenc}
\usepackage{amssymb,amsmath}
\usepackage{float}
\usepackage{lipsum}
\usetikzlibrary{arrows, shapes.gates.logic.US, calc}
\tikzstyle{branch}=[fill, shape=circle, minimum size=3pt, inner sep=0pt]
    
\begin{document}

\section*{Task 1}
NOT gates with BJT transistors (NPN)

a)RTL

*foto del circuito*

b) TTL

*foto del circuito*

\section*{MEASSURES}

Where VIH: minimum HIGH input voltage, VIL: maximum LOW input voltage, VOH: minimum HIGH output voltage, VOL: maximum LOW output voltage.

To meassure these values we use the ramp waveform and the oscilloscope in xy mode so we can see something like this:
*fotis modo xy con puntos marcados*
Where the values we are looking for are found when the derivative is -1.

Noise margin :It allows one to estimate the allowable noise voltage on the input of a gate so that the output will not be affected. Noise margin is specified in terms of two parameters - the low noise margin NL, and the high noise margin NH . NL is defined as the difference in magnitude between the maximum LOW input voltage and the maximum LOW output voltage of the gate. That is, NL =|VIL - VOL|. Similarly, the value of NH is the difference in magnitude between the minimum HIGH output voltage of and the minimum HIGH input voltage recognizable by the gate. That is, NH =|VOH - VIH|. 


The propagation delay  is the difference in time (calculated at 50% of input-output transition), when output switches, after application of input. It is different if the transition is from HIGH to LOW or form LOW to HIGH.

 Rise time is the time, during transition, when output switches from 10% to 90% of the maximum value.
Fall time is the time when output switches from 90% to 10% of the maximum value.


\end{document}