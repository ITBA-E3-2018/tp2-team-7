\newpage
\section*{Task 2}

    In this task we have three different types of NOR gates:

LS family: gates made with TTL technology with Schottky transistors.
HC family: made with CMOS technology, incompatible with TTL circuits.
HCT family: similar to HC but they are compatible with TTL.

Searching for the logic levels of each gate we found that:

\begin{table}[H]
    \begin{center}
        \begin{tabular}{|c||c|c|c|c|}
            \hline 
            Technology & $V_{IL}$ & $V_{IH}$ & $V_{OL}$ & $V_{OH}$\tabularnewline
            \hline 
            \hline 
            TTL & 0.8V & 2V & 0.5V & 2.7V\tabularnewline
            \hline 
            HC (CMOS) & 1.35V & 3.15V & 0.26V & 3.98V\tabularnewline
            \hline 
            HCT (TTL-CMOS) & 0.8V & 2V & 0.33V & 3.84V\tabularnewline
            \hline 
            \end{tabular}
    \caption{Characteristic logical levels at $V_{CC}=4.5V$}
    \end{center}
\end{table}
We can notice that a TTL gate can be connected to the output of a HC gate, but not  otherwise because the minimum output voltage level of the TTL (VOH min) is below the minimum input voltage level CMOS (VIH min) .
In the case of the HCT, as we said before, the logic levels are compatible with TTL gate. 

\section*{Fan out}

Another fundamental aspect that must be taken into account is the maximum current that
provides or absorbs a logic gate. These characteristics are especially
important when the technology of the components is TTL, since bipolar transistors are used and it is necessary to ensure the saturation or cutting state of the
them. Depending on the relationship between maximum output current and
input current, we can connect more or less circuits to a certain logic gate. This maximum number of circuits that I can connect in parallel to the output of a gate is called Fan-Out.
Here the CMOS families present a great advantage thanks to the little
input current (1 uA), which causes the fan-out to be very high. 
On the other hand, if we want to connect TTL gates 
the output capacity will be lower due to the fact that the input current is much higher (1mA).

\section*{Results}

CASE A: LS to HC: from 0,75V to 0,8V there was an invalid output
\newline
CASE B: HC to LS: from 2,05V to 2,17V there was an invalid output
\newline
CASE C: LS to HCT: from 0,8V to 0,9V there was an invalid output
\newline
CASE D: HCT to LS: form 0,85V to 0,91V there was an invalid output
\newline

\subsection*{Case A:}
\newline
The input range where the output was invalid is inside
 the range of voltages allowed for the logic 0 of the LS gate, 
 so the output should be a 0. Here we can see the incompatibility 
 between the gates.
\subsection*{Cases B, C, D:} All the input ranges where the output was invalid
 belong to the invalid input voltages of the gate that is connected 
 first. 

